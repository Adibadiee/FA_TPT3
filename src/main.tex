% !TEX program = lualatex --shell-escape
% !TEX encoding = UTF-8 Unicode

\documentclass[listof=toc, 12pt]{scrartcl}
\usepackage[german]{babel}
\usepackage[a4paper, includefoot, footskip=1.5em, margin=2.5cm]{geometry}
\usepackage[utf8]{inputenc}
\usepackage[style=authoryear, sorting=nyt]{biblatex}
\usepackage{graphicx}
\usepackage{setspace}
\usepackage{hyperref}
\usepackage{sectsty}
\usepackage{fixme}
\usepackage{xcolor}
\usepackage{lmodern}
\usepackage{subcaption}
\usepackage{makecell}
\usepackage{pdfpages}
\usepackage{scrlayer-scrpage}
\usepackage[toc, acronyms]{glossaries}

\fxsetup{status=draft}

%pagenumber on right
\rofoot*{\pagemark}
\cofoot*{}

\renewcommand\multicitedelim{\space{}und\space}
%\addbibresource{main.bib}
%\loadglsentries[acronyms]{acronyms}
\makeglossaries

\setstretch{1.5}

\allsectionsfont{\fontsize{12}{18}\selectfont}
\sectionfont{\fontsize{14}{21}\selectfont}

\setlength{\intextsep}{0pt}

\graphicspath{{Images}}

\hypersetup{hidelinks}

\renewcommand{\footnotesize}{\fontsize{10pt}{10pt}\selectfont}

% \newcommand{\nocontentsline}[3]{}
\newcommand{\tocless}[2]{\bgroup\let\addcontentsline=\nocontentsline#1{#2}\egroup}
\newcommand{\rz}[1]{\glqq{}#1\grqq{}}


\definecolor{BA_Blau}{RGB}{67, 207, 255}


\begin{document}
\pagestyle{empty}


\begin{center}{
        \setstretch{0.9}
        {
            \color{BA_Blau}
            \fontsize{38pt}{46pt}\selectfont % Prüfen ob lineskip korrekt
            Fachpraktische Dokumentation zur Bachelor Thesis mit dem vorläufigem Thema: (THEMA) \\
            \vspace{10pt}
        }
        \vspace{72.8pt} %18pt+8pt+3*1.2*13pt
        \setstretch{1.2}
        {
            \fontsize{14pt}{16.8pt}\selectfont
            Gruppen-Mitglieder (Matrikelnummern):\\
            Aditi Burte   - 231232\\
            ~\\
        }
        \vspace{78pt} %13pt*5*1.2
        {
            \fontsize{14pt}{16.8pt}\selectfont
            Diese fachpraktische Dokumentation wurde erstellt im Rahmen der\\
            \textbf{Theorie-Praxis-Anwendung III}\\
            \vspace{15.6pt} %13pt * 1.2
            Anzahl der Wörter: \\
            \vspace{15.6pt}
            \textbf{Datum: }
        }
    }
\end{center}
\setlength{\intextsep}{12.0pt plus 2.0pt minus 2.0pt}
\newpage
\paragraph{Gender-Hinweis}~\\
In der vorliegenden Ausarbeitung wird darauf verzichtet, bei Personenbezeichnungen sowohl die männliche als auch die weibliche Form zu nennen. Die männliche Form gilt in allen Fällen, in denen dies nicht explizit ausgeschlossen wird, für alle Geschlechter.
\newpage
\paragraph{Sperrvermerk}~\\
Die vorliegende Fachpraktische Ausarbeitung beinhaltet interne vertrauliche
Informationen der DB Systel GmbH. Die Weitergabe des Inhaltes dieser Arbeit und eventuell beiliegender
Abbildungen, Tabellen und Daten im Gesamten oder in Teilen ist grundsätzlich untersagt. Es dürfen
keinerlei Kopien oder Abschriften, auch nicht in digitaler Form, gefertigt werden. Ausnahmen bedürfen
der schriftlichen Genehmigung durch die DB Systel GmbH.
\newpage

\pagestyle{plain}
\pagenumbering{roman}
\setcounter{page}{1}
\tableofcontents
\newpage
\listoffigures
\newpage
\printglossary[nonumberlist, type=\acronymtype, title=Abkürzungsverzeichnis]
\newpage
\newcounter{roman_page_counter}
\setcounter{roman_page_counter}{\value{page}}
\pagenumbering{arabic}
\newpage
\appendix
\renewcommand{\thesection}{A\arabic{section}}
\pagenumbering{roman}
\setcounter{page}{\value{roman_page_counter}}
\printbibliography[heading=bibintoc]

\end{document}
