% !TEX program = lualatex --shell-escape
% !TEX encoding = UTF-8 Unicode

\documentclass[listof=toc, 12pt]{scrartcl}
\usepackage[german]{babel}
\usepackage[a4paper, includefoot, footskip=1.5em, margin=2.5cm]{geometry}
\usepackage[utf8]{inputenc}
\usepackage[style=authoryear, sorting=nyt]{biblatex}
\usepackage{graphicx}
\usepackage{setspace}
\usepackage{hyperref}
\usepackage{sectsty}
\usepackage{fixme}
\usepackage{xcolor}
\usepackage{lmodern}
\usepackage{subcaption}
\usepackage{makecell}
\usepackage{pdfpages}
\usepackage{scrlayer-scrpage}
\usepackage[toc, acronyms]{glossaries}

\fxsetup{status=draft}

%pagenumber on right
\rofoot*{\pagemark}
\cofoot*{}

\renewcommand\multicitedelim{\space{}und\space}
%\addbibresource{main.bib}
%\loadglsentries[acronyms]{acronyms}
\makeglossaries

\setstretch{1.5}

\allsectionsfont{\fontsize{12}{18}\selectfont}
\sectionfont{\fontsize{14}{21}\selectfont}

\setlength{\intextsep}{0pt}

\graphicspath{{Images}}

\hypersetup{hidelinks}

\renewcommand{\footnotesize}{\fontsize{10pt}{10pt}\selectfont}

% \newcommand{\nocontentsline}[3]{}
\newcommand{\tocless}[2]{\bgroup\let\addcontentsline=\nocontentsline#1{#2}\egroup}
\newcommand{\rz}[1]{\glqq{}#1\grqq{}}


\definecolor{BA_Blau}{RGB}{67, 207, 255}


\begin{document}
\pagestyle{empty}


\begin{center}{
        \setstretch{0.9}
        {
            \color{BA_Blau}
            \fontsize{38pt}{46pt}\selectfont % Prüfen ob lineskip korrekt
            Fachpraktische Dokumentation zur Bachelor Thesis mit dem vorläufigem Thema: Evaluation of Dashboard Tools - Critical review and classification of tool-independent criteria for tool comparison at DB Systel GmbH. \\
            \vspace{10pt}
        }
        \vspace{72.8pt} %18pt+8pt+3*1.2*13pt
        \setstretch{1.2}
        {
            \fontsize{14pt}{16.8pt}\selectfont
            Gruppen-Mitglieder (Matrikelnummern):\\
            Aditi Burte   - 231232\\
            ~\\
        }
        \vspace{78pt} %13pt*5*1.2
        {
            \fontsize{14pt}{16.8pt}\selectfont
            Diese fachpraktische Dokumentation wurde erstellt im Rahmen der\\
            \textbf{Theorie-Praxis-Anwendung III}\\
            \vspace{15.6pt} %13pt * 1.2
            Anzahl der Wörter: \\
            \vspace{15.6pt}
            \textbf{Datum: }
        }
    }
\end{center}
\setlength{\intextsep}{12.0pt plus 2.0pt minus 2.0pt}
\newpage
\paragraph{Gender-Hinweis}~\\
In der vorliegenden Ausarbeitung wird darauf verzichtet, bei Personenbezeichnungen sowohl die männliche als auch die weibliche Form zu nennen. Die männliche Form gilt in allen Fällen, in denen dies nicht explizit ausgeschlossen wird, für alle Geschlechter.
\newpage
\paragraph{Sperrvermerk}~\\
Die vorliegende Fachpraktische Ausarbeitung beinhaltet interne vertrauliche
Informationen der DB Systel GmbH. Die Weitergabe des Inhaltes dieser Arbeit und eventuell beiliegender
Abbildungen, Tabellen und Daten im Gesamten oder in Teilen ist grundsätzlich untersagt. Es dürfen
keinerlei Kopien oder Abschriften, auch nicht in digitaler Form, gefertigt werden. Ausnahmen bedürfen
der schriftlichen Genehmigung durch die DB Systel GmbH.
\newpage

\pagestyle{plain}
\pagenumbering{roman}
\setcounter{page}{1}
\tableofcontents
\newpage
\listoffigures
\newpage
\printglossary[nonumberlist, type=\acronymtype, title=Abkürzungsverzeichnis]
\newpage
\newcounter{roman_page_counter}
\setcounter{roman_page_counter}{\value{page}}
\pagenumbering{arabic}
\newpage
\appendix
\subsection{Problemidentifikation} ~\\ 
Dashboard-Tools haben sich als zentrale Elemente der modernen Business Intelligence etabliert. Sie transformieren komplexe Daten in visuell zugängliche und interaktive Erkenntnisse, die fundierte Entscheidungen auf allen Unternehmensebenen ermöglichen. Eine effektive Datenvisualisierung ist somit nicht nur ein technisches Hilfsmittel, sondern ein strategischer Wettbewerbsfaktor, der Agilität und datengesteuerte Prozesse fördert.

Im Rahmen des dualen Studiums, insbesondere während der Praxisphase bei der DB Systel GmbH, ergab sich eine vertiefte Auseinandersetzung mit der Thematik der Datenvisualisierung. Dabei hatte ich die Gelegenheit verschiedene Tools wie Power BI, Tableau sowie Programmierbibliotheken wie Matplotlib, Pandas und HvPlot kennenzulernen. Hierbei fiel mir auf, dass innerhalb des Unternehmens keine einheitliche Vorgehensweise für die Auswahl der Tools existiert. Jedes Team entscheidet eigenständig, oft basierend auf individuellen Vorlieben, bestehendem Know-how oder kurzfristigen Projektzielen[AB3.1]. Diese dezentrale Herangehensweise führt zu einer uneinheitlichen Tool-Landschaft und erschwert die Zusammenarbeit zwischen den Teams. Sie verursacht außerdem eine heterogene Ausgangslage bei der Entwicklung und dem Einsatz von Dashboards im gesamten \\

\subsection{Problembeschreibung}~ \\
Das zentrale Problem liegt in einem dezentralen und intransparenten Entscheidungsprozess zur Auswahl von Dashboard-Tools bei der DB Systel. Jedes Team trifft seine Wahl eigenständig und ohne übergreifende Abstimmung oder Zugriff auf vorhandenes Wissen innerhalb des Unternehmens. Dadurch fehlt eine strategische Gesamtperspektive, die sicherstellen würde, dass Entscheidungen auf Basis objektiver Kriterien und bestehender Erfahrungen getroffen werden. Häufig beginnen Teams ihre Entwicklungen mit einem Tool, um später festzustellen, dass ein anderes Werkzeug – etwa im Hinblick auf Performance bei Echtzeitdaten, Integrationsfähigkeit oder spezialisierte Visualisierungsfunktionen – besser für den jeweiligen Use Case geeignet gewesen wäre. Diese Fehlentscheidungen entstehen direkt aus dem mangelnden Wissensaustausch und der fehlenden Möglichkeit, vorhandene Lösungen oder Erfahrungswerte anderer Teams zu berücksichtigen.
Das Ergebnis ist eine fragmentierte und ineffiziente Tool-Landschaft: Ähnliche Dashboards werden parallel in unterschiedlichen Tools entwickelt, wodurch doppelte Arbeit entsteht. Diese Redundanzen    führen nicht nur zu erhöhtem Entwicklungsaufwand, sondern auch zu langfristigen Mehrkosten durch zusätzlichen Wartungs-, Lizenz- und Schulungsbedarf. Darüber hinaus erschwert die technologische Vielfalt die Integration und den Wissenstransfer zwischen den Teams, was letztlich Innovation hemmt, und die Skalierbarkeit der Lösungen einschränkt. Ein unternehmensweiter, standardisierter Bewertungsrahmen könnte hier Abhilfe schaffen, indem er den Entscheidungsprozess strukturiert, Wissensaustausch ermöglicht und die Ressourcen effizienter lenkt\\

\subsection{Vorgehensweise und Maßnahmen}~ \\ 
Um dieses Problem zu lösen, soll ein strukturiertes, mehrstufiges Vorgehen entwickelt werden. In einem ersten Schritt werden durch Literaturrecherchen sowie Gespräche mit internen Fachbereichen relevante funktionale und nicht-funktionale Kriterien ermittelt. Diese dienen als Grundlage für einen einheitlichen Kriterienkatalog zur Bewertung von Dashboard-Tools. Im nächsten Schritt werden die Kriterien thematisch geordnet und gemeinsam mit den Stakeholdern gewichtet, um ihre Bedeutung für den Unternehmenskontext zu bestimmen. In der letzten Phase wird der Katalog auf ausgewählte Tools angewendet. Mithilfe von praktischen Tests und bahnspezifischen Use-Cases wird überprüft, wie gut die einzelnen Tools geeignet sind. Das Ergebnis dieser Arbeit ist eine Bewertungsmatrix, die eine transparente und objektive Entscheidungsgrundlage für zukünftige Projektauswahlen bietet. Sie schafft gleichzeitig die Basis für die spätere Bachelorarbeit, in der untersucht werden soll, welches Tool für welchen Use-Case am besten geeignet ist.\\
\renewcommand{\thesection}{A\arabic{section}}
\pagenumbering{roman}
\setcounter{page}{\value{roman_page_counter}}
\printbibliography[heading=bibintoc]

\end{document}
