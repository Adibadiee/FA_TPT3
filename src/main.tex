% !TEX program = lualatex --shell-escape
% !TEX encoding = UTF-8 Unicode

\documentclass[listof=toc, 12pt]{scrartcl}
\usepackage[ngerman]{babel}
\usepackage[utf8]{inputenc}
\usepackage{csquotes}
\usepackage[a4paper, includefoot, footskip=1.5em, margin=2.5cm]{geometry}
\usepackage[style=authoryear, sorting=nyt]{biblatex}
\usepackage{graphicx}
\usepackage{setspace}
\usepackage{hyperref}
\usepackage{sectsty}
\usepackage{fixme}
\usepackage{xcolor}
\usepackage{lmodern}
\usepackage{subcaption}
\usepackage{makecell}
\usepackage{pdfpages}
\usepackage[singlespacing=true]{scrlayer-scrpage}
\usepackage[toc, acronyms]{glossaries}

\fxsetup{status=draft}

%pagenumber on right
\rofoot*{\pagemark}
\cofoot*{}

\renewcommand\multicitedelim{\space{}und\space}
%\addbibresource{main.bib}
%\loadglsentries[acronyms]{acronyms}
\makeglossaries

\setstretch{1.5}

\allsectionsfont{\fontfamily{cmr}\selectfont\fontsize{12}{18}\selectfont}
\sectionfont{\fontfamily{cmr}\selectfont\fontsize{14}{21}\selectfont}

\setlength{\intextsep}{0pt}

\graphicspath{{Images/}}

\hypersetup{hidelinks}

\renewcommand{\footnotesize}{\fontsize{10pt}{10pt}\selectfont}

% \newcommand{\nocontentsline}[3]{}
\newcommand{\tocless}[2]{\bgroup\let\addcontentsline=\nocontentsline#1{#2}\egroup}
\newcommand{\rz}[1]{\glqq{}#1\grqq{}}


\definecolor{BA_Blau}{RGB}{0, 0, 148}


\begin{document}
\pagestyle{empty}


\begin{center}{
        \setstretch{0.9}
        {
            \color{BA_Blau}
            \fontsize{38pt}{46pt}\selectfont % Prüfen ob lineskip korrekt
            Fachpraktische Dokumentation zur Bachelor Thesis mit dem vorläufigem Thema: Evaluation of Dashboard Tools - Critical review and classification of tool-independent criteria for tool comparison at DB Systel GmbH. \\
            \vspace{10pt}
        }
        \vspace{65pt} %18pt+8pt+3*1.2*13pt
        \setstretch{1.2}
        {
            \fontsize{14pt}{16.8pt}\selectfont
            \textbf{Aditi Burte (231232)}\\
            ~\\
        }
        \vspace{70pt} %13pt*5*1.2
        {
            \fontsize{14pt}{16.8pt}\selectfont
            Diese fachpraktische Dokumentation wurde erstellt im Rahmen der\\
            \textbf{Theorie-Praxis-Anwendung III}\\
            \vspace{15.6pt} %13pt * 1.2
            Anzahl der Wörter: 511\\
            \vspace{15.6pt}
            \textbf{Datum: 17. November 2025 }
        }
    }
\end{center}
\setlength{\intextsep}{12.0pt plus 2.0pt minus 2.0pt}
\newpage
\pagestyle{plain}
\pagenumbering{roman}
\setcounter{page}{1}
\newcounter{roman_page_counter}
\setcounter{roman_page_counter}{\value{page}}
\pagenumbering{arabic}
\newpage
\printglossary[nonumberlist, type=\acronymtype, title=Abkürzungsverzeichnis]

\paragraph{Einleitung}~\\
Dashboard-Tools haben sich als zentrale Elemente der modernen Business-Intelligence etabliert. Sie transformieren komplexe Daten in interaktive Visualisierungen, die fundierte Entscheidungen auf allen Unternehmensebenen ermöglichen. Eine effektive und effiziente Datenvisualisierung ist nicht nur ein technisches Hilfsmittel, sondern fördert auch Agilität und datengesteuerte Prozesse.

\newpage
\paragraph{Problemidentifikation}~\\ 
Im Rahmen meiner letzten Praxisphasen bei der DB Systel GmbH ergab sich eine vertiefte Auseinandersetzung mit der Thematik der Datenvisualisierung. Dabei hatte ich die Gelegenheit verschiedene Tools wie Power BI, Tableau sowie Programmierbibliotheken wie Matplotlib, Pandas und HvPlot kennenzulernen. Hierbei fiel mir auf, dass innerhalb des Unternehmens keine einheitliche Vorgehensweise für die Auswahl der Tools existiert. Deswegen sind die Entscheidungsprozesse uneinheitlich und schwer nachvollziehbar. Daraus ergibt sich die Frage, wie sich dieser Auswahlprozess strukturieren lässt, um eine fundierte vergleichbare Entscheidung zu ermöglichen.

\newpage
\paragraph{Problembeschreibung}~\\
Das Kernproblem besteht in einem dezentralen und intransparenten Verfahren zur Auswahl von Dashboard-Tools bei der DB System GmbH. 
Im Rahmen des Moduls Geschäfts\-prozess\-management hatten wir uns thematisch mit Entscheidungs\-prozessen auseinandergesetzt. Damit ist ein strukturiertes Vorgehen gemeint, bei dem Entscheidungen schrittweise auf der Grundlage klarer Kriterien und verfügbarer Informationen getroffen werden. Schließlich ermöglicht es, dass die Resultate zielgerichtet sind. Die einzelnen Teams treffen ihre Entscheidungen unabhängig voneinander, da es keine unternehmensweite Abstimmung oder zentrale Wissensbasis gibt. Diese isolierte Vorgehensweise führt zu einer heterogenen Ausgangslage, in der verschiedene Teams mit unterschiedlichen technischen Voraussetzungen und Toolsets starten. Die zeigt, dass es an einer strategischen Gesamtperspektive mangelt und Entscheidungen häufig nicht auf objektiv vergleichbaren Kriterien oder bestehenden Erfahrungen beruhen. Eine Folge hiervon ist, dass Teams ihre Entwicklungen zum Teil mit Tools starten, die sich später als ungeeignet herausstellen. Zum Beispiel, weil sie Einschränkungen hinsichtlich der Performance, der Datenverarbeitung in Echtzeit oder der Visualisierungsmöglichkeiten nicht betrachtet haben. Die fehlende Vergleichsgrundlage bei der Tool-Auswahl trägt zur Entstehung einer fragmentierten und ineffektiven Tool-Landschaft bei. Darüber hinaus werden ähnliche Dashboards parallel in unterschiedlichen Systemen erstellt, was zur doppelten Entwicklungsarbeit führt. Die vorliegenden Inkonsistenzen erschweren nicht nur die Vergleichbarkeit der Ergebnisse, sondern erhöhen auch den Abstimmungs- und Wartungsaufwand zwischen den Teams. Langfristig entsteht dadurch ein ineffizienter Ressourceneinsatz mit hohen Kosten für Entwicklung, Pflege und Schulung. Zudem wird der Wissenstransfer beeinträchtigt, da Erfahrungen aufgrund der unterschiedlichen Technologien kaum geteilt werden können. Ein unternehmensweiter, standardisierter Bewertungsrahmen könnte in diesem Kontext eine Lösung bieten, indem er den Entscheidungsprozess strukturiert, Vergleichbarkeit ermöglicht und vorhandenes Wissen systematisch nutzbar macht.

\newpage
\paragraph{Vorgehensweise und Maßnahmen}~\\
Zusammenfassend lässt sich sagen, dass der unstrukturierte Entscheidungsprozess negative Auswirkungen auf die DB Systel GmbH hat.
Zur Lösung des identifizierten Problems wird die Entwicklung eines standardisierten Bewertungsrahmens vorgeschlagen. Außerdem ist es notwendig einem mehrstufigen Verfahren zu folgen. Hierfür werden schrittweise durch Literaturrecherchen sowie Gespräche mit internen Fachbereichen relevante funktionale und nicht-funktionale Kriterien ermittelt. Diese können anschließend thematisch geordnet und gewichtet werden. Sie bilden die Grundlage für einen einheitlichen Kriterienkatalog zur Bewertung von Dashboard-Tools. 
\newpage
\appendix
\renewcommand{\thesection}{A\arabic{section}}
\pagenumbering{roman}
\setcounter{page}{\value{roman_page_counter}}
\printbibliography[heading=bibintoc]
\pagestyle{empty}
\paragraph{Eidesstattliche Erklärung}~\\
Hiermit erkläre ich, dass ich die vorliegende fachpraktische Ausarbeitung selbstständig verfasst und keine anderen als die angegebenen Quellen und Hilfsmittel benutzt und die aus fremden Quellen direkt oder indirekt übernommenen Gedanken als solche kenntlich gemacht habe. Die Arbeit oder Teile hieraus wurde und wird keiner anderen Stelle oder anderen Person im Rahmen einer Prüfung vorgelegt. Ich versichere zudem, dass keine sachliche Übereinstimmung mit einer im Rahmen eines vorangegangenen Studiums angefertigten Seminar-, Haus-, Diplom- oder Abschlussarbeit sowie Bachelor-Thesis besteht.
\newpage


\end{document}
